% This is file miirguide2.tex
% release v1.0, 1 January 2021
%   (based on ejm2egui.tex v 1.02, JFM2egui.tex v1.13 for LaTeX 2e
%         and EJMguide.tex v0.2 for LaTeX 2.09)
% Copyright (C) 1998,1999,2000, 2021 Cambridge University Press

\NeedsTeXFormat{LaTeX2e}

% The following saves the original definitions of \geq and \leq (guide only).
\let\realgeq\geq
\let\realleq\leq

\documentclass{MIIR}
\usepackage{amsmath,amsthm}
\usepackage{graphicx}
\usepackage{floatpag}
\usepackage{float}
\usepackage{xcolor}
\usepackage[colorlinks=true,linkcolor=blue]{hyperref}

\usepackage{tikz,pgfplots} %required for tikz plots
\pgfplotsset{compat=1.3}
\usetikzlibrary{external}
\tikzexternalize[prefix=tikz/]
\usepackage[numbers]{natbib}

%%% Macros for the guide only %%%
\newcommand\lra{\ensuremath{\quad\longrightarrow\quad}}
\newcommand\rch[1]{\lra\rlap{$#1$}\hspace{1em}}

% The following adds 6pt of space around verbatim environments.
\let\realverbatim=\verbatim
\let\realendverbatim=\endverbatim
\renewcommand\verbatim{\par\addvspace{6pt plus 2pt minus 1pt}\realverbatim}
\renewcommand\endverbatim{\realendverbatim\addvspace{6pt plus 2pt minus 1pt}}
%%% End of macros for the guide %%%





%%% Example macros (some are not used in this sample file) %%%

% For units of measure
\newcommand\dynpercm{\nobreak\mbox{$\;$dynes\,cm$^{-1}$}}
\newcommand\cmpermin{\nobreak\mbox{$\;$cm\,min$^{-1}$}}

% Various bold symbols
\providecommand\bnabla{\boldsymbol{\nabla}}
\providecommand\bcdot{\boldsymbol{\cdot}}
\providecommand\bomega{\boldsymbol{\omega}}
\newcommand\biS{\boldsymbol{S}}
\newcommand\etb{\boldsymbol{\eta}}
\newcommand\bldP{\boldsymbol{P}}

% For multiletter symbols
\newcommand\Real{\mbox{Re}} % cf plain TeX's \Re and Reynolds number
\newcommand\Imag{\mbox{Im}} % cf plain TeX's \Im
\newcommand\Rey{\mbox{\textit{Re}}}  % Reynolds number
\newcommand\Pran{\mbox{\textit{Pr}}} % Prandtl number, cf TeX's \Pr product
\newcommand\Pen{\mbox{\textit{Pe}}}  % Peclet number
\newcommand\Ai{\mbox{Ai}}            % Airy function
\newcommand\Bi{\mbox{Bi}}            % Airy function


\newcommand\ssC{\mathsf{C}}    % for sans serif C
\newcommand\sfsP{\mathsfi{P}}  % for sans serif sloping P
\newcommand\slsQ{\mathsfbi{Q}} % for sans serif bold-sloping Q

% Hat position
\newcommand\hatp{\skew3\hat{p}}      % p with hat
\newcommand\hatR{\skew3\hat{R}}      % R with hat
\newcommand\hatRR{\skew3\hat{\hatR}} % R with 2 hats
\newcommand\doubletildesigma{\skew2\tilde{\skew2\tilde{\Sigma}}}
%       italic Sigma with double tilde

% array strut to make delimiters come out right size both ends
\newsavebox{\astrutbox}
\sbox{\astrutbox}{\rule[-5pt]{0pt}{20pt}}
\newcommand{\astrut}{\usebox{\astrutbox}}

\newcommand\GaPQ{\ensuremath{G_a(P,Q)}}
\newcommand\GsPQ{\ensuremath{G_s(P,Q)}}
\newcommand\p{\ensuremath{\partial}}
\newcommand\tti{\ensuremath{\rightarrow\infty}}
\newcommand\kgd{\ensuremath{k\gamma d}}
\newcommand\shalf{\ensuremath{{\scriptstyle\frac{1}{2}}}}
\newcommand\sh{\ensuremath{^{\shalf}}}
\newcommand\smh{\ensuremath{^{-\shalf}}}
\newcommand\squart{\ensuremath{{\textstyle\frac{1}{4}}}}
\newcommand\thalf{\ensuremath{{\textstyle\frac{1}{2}}}}
\newcommand\Gat{\ensuremath{\widetilde{G_a}}}
\newcommand\ttz{\ensuremath{\rightarrow 0}}
\newcommand\ndq{\ensuremath{\frac{\mbox{$\partial$}}{\mbox{$\partial$} n_q}}}
\newcommand\sumjm{\ensuremath{\sum_{j=1}^{M}}}
\newcommand\pvi{\ensuremath{\int_0^{\infty}}}%

\newcommand\etal{\mbox{\textit{et al.}}}
\newcommand\eg{e.g.\ }
\providecommand\AMSLaTeX{AMS\,\LaTeX}

\theoremstyle{plain}
\newtheorem{theorem}{Theorem}[section]   %%number by section
\newtheorem{lemma}[theorem]{Lemma}
\newtheorem*{corollary*}{Corollary}  %% unnumbered

 \theoremstyle{definition}
 \newtheorem{definition}[theorem]{Definition}
 \newtheorem{problem}[theorem]{Problem}
 \newtheorem{example}[theorem]{Example}
 \newtheorem*{problem*}{Problem}    %% unnumbered
 \newtheorem*{example*}{Example}
  
  
\title[Guide for Mathematics in Industry Reports]{\LaTeXe\ %
         Guide for Mathematics in Industry Reports}


\author[Tranah et al.]{%
  D.A. Tranah$\,^1$\thanks{Corresponding Author: {\tt dtranah@cambridge.org}},\ls
  W. Weierstrass$\,^2$\thanks{Thanks to S. Schmidt for assistance}\ls
\and
  Z. Zhang$\,^3$\thanks{Now at Engines Plc, Beihai, China}
}

\affiliation{%
  $^1\,$Cambridge University Press, Cambridge, UK\\
    $^2\,$University of Harford, Harford, USA\\
  $^3\,$Allgemeine Industrials, Frankburg, Germany}
  
\date{\today}
\pubyear{2021}
\issue{\SG}
\pagerange{\pageref{firstpage}--\pageref{lastpage}}

%\acceptedtrue

\begin{document}

\label{firstpage}
\maketitle

\begin{footnotesize}

\begin{studygroup}
This is the name of the Study Group. Include identifier (for example ESGI and number), series name (if any), location and date.
\end{studygroup}

\begin{communicated}
Give here the name of the Study Group organiser.
\end{communicated}

\begin{partner}
Give the name here of the industrial or business partner and a url if required.
\end{partner}

\begin{presenter}
Include the name of the individual(s) who proposed the problem, and address if required.
\end{presenter}


\begin{team}
List here all team members in alphabetical order who contributed to the solution of the problem, e.g.\\
A.N. Contributor1, Affiliation1;
A.N. Contributor2, Affiliation2;
A.N. Contributor3, Affiliation3.\\
Team members are not necessarily the same as authors. See \S\ref{authors}.
\end{team}

\begin{application}
Include the main industrial/business application area from one of the following categories (also available from the \href{https://www.cambridge.org/engage/miir/public-dashboard}{MIIR} site):
Aerospace;
Agriculture/Fisheries;
Biomedical/Healthcare;
Charities;
Chemical;
Communications/Networks;
Computing/Robotics;
Construction;
Data Analysis;
Defence;
Electronics;
Energy/Utilities;
Environment;
Finance;
Food and Drink;
Government;
Logistics;
Manufacturing;
Materials processing;
Mechanics;
Pharmaceutical;
Retail;
Social;
Sports;
Textile/Clothing/Footwear;
Transport.
\end{application}

\begin{tools}
Include the main mathematical tools and models used to tackle the problem.
\end{tools}


\begin{keywords}
Include up to five `key subject categories', listed in order of importance.
\end{keywords}


\begin{MSC2020}
Include up to five two-level subject codes, listed in order of importance, taken from the Mathematics Subject Classification
(see \href{https://mathscinet.ams.org/msnhtml/msc2020.pdf}{MSC2020}).
\end{MSC2020}
\end{footnotesize}
%\clearpage




\maketitle

\summary{This guide is for authors who are preparing submissions for the
\emph{Mathematics in Industry Reports}
using the \LaTeXe\ document preparation system and the Cambridge
University Press MIIR class file. It should be used as a template for your report.}


\section{Introduction}

The layout for \emph{Mathematics in Industry Reports} has been
implemented as a \LaTeXe\ class file. The MIIR class is based on the article.cls.
Commands which differ from  standard \LaTeX, or which are
provided in addition to it, are explained in this
guide. This guide is not a substitute for the \LaTeX\ manual itself.

Please do not call in unnecessary packages. Not only is it bad style,  but they may interfere with
packages that are required in the class file.


\subsection{General style issues}

Use of \LaTeX\ defaults will result in a pleasing uniformity of layout
and font selection. Authors should resist the temptation to make
\emph{ad hoc} changes to these. Also avoid use of direct formatting unless
really necessary. 

If you have written your report using article.cls, then remove hard breaks before converting to MIIR.cls
as the page and line lengths are different.

The language used 
can be either British or American English, provided it's consistent within each report.

Enunciations (i.e. Theorems, examples, etc.), should you wish them, are handled using amsthm.sty. You therefore need to call in both amsmath.sty and amsthm.sty, \emph{in that order} in the root file, as they are not called by the class file. 
See \S\ref{sectTheor}.
We recommend calling  hyperref.sty in your root file to enable live links within a document. Add a \verb"\label" to anything that you refer to and use \verb"\ref",  \verb"\href" and \verb"\cite"
commands when referencing or citing  items. See \S\ref{refs}.
To keep things simple we recommend using natbib.sty to handle reference lists and citations. Include the command \verb"\usepackage[numbers]{natbib}" in the root file so that everything is numbered.

The MIIR.cls file specifies how other things, such as sections or equations, will be numbered. 




\subsection{Submission of Reports}

Authors who intend to submit a report should obtain a copy of the
MIIR class file from the submission site.
You will also find there this guide; you can use it as a template for your report.

You should submit a pdf of the final version of your report created in \LaTeX\ using the MIIR class file.



\section{Using the MIIR class file}

First, copy the file \verb"MIIR.cls"  into the correct subdirectory on your system.
\verb"MIIR.cls"  is a complete document class, and
\emph{not} a class option; thus,
if you're familiar with the standard article class then simply replace \verb"article"  with \verb"MIIR" in the
\verb"\documentclass" command at the beginning of your report. Do \emph{not} include any optional arguments that specify the font size.


Author-defined macros should be inserted before \verb"\begin{document}".
You\emph{ must not} change any of the macro definitions
or parameters in \verb"MIIR.cls".

\subsection{Document class options}\label{sec:ClassOp}

In general, the following standard document class options should \emph{not} be
used with \verb"MIIR.cls":
%
\begin{itemize}
  \item \texttt{10pt}, \texttt{11pt} and \texttt{12pt} -- unavailable;
  \item \texttt{twoside} is the default (\texttt{oneside} is disabled);
  \item \texttt{onecolumn} is the default (\texttt{twocolumn} is disabled);
  \item \texttt{titlepage} is not required and is disabled; \verb"\maketitle" creates the title page;
  \item \texttt{abstract} is not required and has been disabled. See \S\ref{abstract} below.
  \item \texttt{fleqn} and \texttt{leqno} should not be used, and are disabled.
\end{itemize}

\subsection{Additional packages required for \texttt{MIIR.cls}}
\label{sec:adpkg}
The following additional package (i.e., \verb".sty" files) are required but are not supplied: they can
be loaded from your local \LaTeX\ installation or downloaded from the eb, e.g. from a CTAN site.

\begin{itemize}
    \item \verb"amsmath, amsfonts, amsthm" -- required for standard fonts and environments.
    \item \verb"natbib" -- required for reference lists and citations to it.
    \item \verb"graphicx, floatpag, float" -- required for figures and floats.
    \item \verb"hyperref" is not required but is strongly recommended as it menas you can create internal and external links.
\end{itemize}


\section{Extras}

In addition to all of the standard \LaTeX\ design elements, \verb"MIIR.cls"  
includes the following features:
\begin{itemize}
  \item Extended commands for specifying a short version
        of the title and author(s) for the running
        headlines.
  \item Control of enumerated lists.
  \item Various `metadata' environments: {\tt communicated}, {\tt keywords}, {\tt application}, {\tt MSC2020}, {\tt studygroup}, {\tt presenter}, {\tt partner}, {\tt team}, {\tt tools}, all of which are illustrated on page 1 of this guide.
  \item A {\tt summary} environment, which should begin the report proper.
  \item An {\tt acknowledgments} environment.
  \item Extensions of the {\tt proof} environment.
\end{itemize}
Once you have used any of these additional facilities in your document,
it can only be processed with \verb"MIIR.cls".

\section{Title page}
\subsection{Title}
In the MIIR style, the title of the report and the author name(s)
are used both at the beginning of the report for the main title and
throughout the report as running heads at the top of every page.
The title, or a short version of it, is used on odd-numbered pages (rectos) and the author name
appears on even-numbered pages
(versos). The \verb"\pagestyle" and \verb"\thispagestyle" commands should
\emph{not} be used.  Similarly, the commands \verb"\markright" and
\verb"\markboth" should not be necessary.

Although the report title can run to several lines of text,
the running head must be a single line.
Additionally, the title can also incorporate kine break commands
(\eg \verb"\\") but these are not acceptable in a running head.
To enable you to specify an alternative short title, the
standard \verb"\title" 
%and \verb"\author"
 command has been extended to take
an optional argument to be used as the running head:
%
\begin{verbatim}
  \title[A short title]{The full title which can be as long as necessary}
\end{verbatim}


\subsection{Authors and Team}\label{authors}

\paragraph{Authors}
Only those people who have contributed to the actual writing of the report should be listed as authors. 

As with the recto running head, the verso must also be only a single line. Therefore the \verb"\author" command also includes an optional argument which should just consist of the family name(s), like so:

\begin{verbatim}
 \author[Last names of authors]{The full names of all the authors, 
               using letterspacing (the command \ls) and  
               \and before the final name in the list}
\end{verbatim}


Note: If there are three or more authors of your report, the optional argument should contain the first author's name followed by `et al.'.

Names will automatically appear in small caps, so only use upper case letters for initials and the first  letter of the family name(s).

Author affiliations must be entered using
the \verb"\affiliation" command as shown below.   

The following example shows how. Note that you have to insert the affiliation footnote
numbers manually. 

%
\begin{verbatim}
\author[Tranah et al.]{%
  D.A. Tranah$\,^1$\thanks{Corresponding Author: {\tt dtranah@cambridge.org}},\ls
  W. Weierstrass$\,^2$\thanks{Thanks to S. Schmidt for assistance}\ls
\and
  Z. Zhang$\,^3$\thanks{Now at Engines Plc, Hamstadt, Germany}
}

\affiliation{%
  $^1\,$Cambridge University Press, Cambridge, UK\\
    $^2\,$University of Harford, Harford, USA\\
  $^3\,$Allgemeine Industrials, Frankburg, Germany}
  \end{verbatim}
%
The \LaTeX\ \verb"\thanks" command, to be used inside the \verb"\author" command, produced here the detials of the coresponding author, 
a personal acknoweldgement and an author's `current address'  as footnotes on the title page, but it could also be used for other details, as appropriate.

\paragraph{Team}
The full list of those contributing to the study group problem should be given within the `team' environment (see below).

\subsection{Other information}

Also required, as appropriate, on the title page is information about the Study Group (the `studygroup' environment),  the Study Group organiser responsible for the report (`communicated'), the name of the industrial or business partner (`partner'), the name(s) of the individual(s) who contributed the problem (`presenter'), the names of members of the team who worked on the problem addressed in the report (`team'), the industrial application area (`application'), tools and models used to tackle the problem (`tools'),  keywords (`keywords'), and 
MSC2020 codes (`MSC2020'), in that order.  Explicitly:

\begin{verbatim}

\begin{studygroup}
...
\end{studygroup}

\begin{communicated}
...
\end{communicated}

\begin{partner}
...
\end{partner}

\begin{presenter}
...
\end{presenter}

\begin{team}
...
\end{team}

\begin{application}
...
\end{application}

\begin{tools}
...
\end{tools}

\begin{keywords}
...
\end{keywords}

\begin{MSC2020}
...
\end{MSC2020}
\end{verbatim}


\subsection{Summary}\label{abstract}
Please include at the start of the report a summary, no longer than 150 words,
 using the  \verb"\summary" command, like so:

\begin{verbatim}
\summary{This is a summary of the paper in less than 150 words.}
\end{verbatim}

You will need to use the same text in the `Abstract' box on the submission page.


\section{Lists}

The MIIR style provides the three standard list environments:
\begin{itemize}
  \item Numbered lists, created using the \verb"enumerate" environment.
  \item Bulleted lists, created using the \verb"itemize" environment.
  \item Labelled lists, created using the \verb"description" environment.
\end{itemize}
The \verb"enumerate" environment numbers each list item with an arabic
number in parentheses; alternative styles can be achieved by inserting
a redefinition of the number labelling command after the
\verb"\begin{enumerate}". For example, a list numbered with roman numerals
inside parentheses can be produced by the following commands:
%
\begin{verbatim}
  \begin{enumerate}[(iii)]
  \renewcommand{\theenumi}{\roman{enumi}}
  \item first item
         :
  \end{enumerate}
\end{verbatim}
The optional argument contains the widest `number' so that alignment of the numbers is correct. 


\section{Enunciation and and Proof environments}\label{sectTheor}

\subsection{Enunciations}

The amsthm package provides a standard way of allowing varying types of theorem-like enunciations to
be laid out differently but consistently, and to be numbered automatically within
a numbering system of your choice; and it's easy to implement. We don't include it within the .cls file, but
suggest you call it
include near top of the root file the following lines, for example:

\verb"  \documentclass{"\texttt{MIIR}\verb"}"\\
\verb"  \usepackage{amsmath}"\\
\verb"  \usepackage{amsthm}"\\[0.5\baselineskip]
Note that \verb"amsmath.sty" \emph{must} precede \verb"amsthm.sty".

Layout of enunciations is determined by the \verb"\theoremstyle" command
given in the preamble of the root file. If no style is specified, it will default to \texttt{plain}.
To specify a different style (we only recommend plain and definition styles),
divide your \verb"\newtheorem" commands
into two groups and preface each one with the appropriate \verb"\theoremstyle".
Enunciations can be numbered or unnumbered.

\paragraph{amsthm \texttt{plain} style}
The \texttt{plain} style is normally used for theorems, lemmas,
corollaries, propositions, and conjectures. These can be numbered or unnumbered. The text of the enunciation will be typeset in italic.

\paragraph{amsthm \texttt{definition} style}
The \texttt{definition} style is used for definitions,
remarks, notation, conditions, problems, examples, problems, projects or other such items that might need to be set off and numbered. The text of the enunciation will be typeset in roman.


Our preferred style is that enunciations should be numbered in a single
sequence either by report (thus, Definition~1, Lemma~2,
Example~3), or by section (thus, Definition~1.1, Lemma~1.2, 
Example~2.1): this helps navigation. 
It's good style only to number things that are refered to. 

The example below illustrates the use of both styles, in numbered and unnumbered form.
The following code sets up the style and the numbering system. Here's how to number by section.

\begin{verbatim}
  \theoremstyle{plain}% default
  \newtheorem{theorem}{Theorem}[section]  %%number by section
  \newtheorem{lemma}[theorem]{Lemma}
  \newtheorem*{corollary*}{Corollary}  %% unnumbered

  \theoremstyle{definition}
  \newtheorem{definition}[theorem]{Definition}
 \newtheorem{problem}[theorem]{Problem}
  \newtheorem{example}[theorem]{Example}
  \newtheorem*{problem*}{Problem}    %% unnumbered
  \newtheorem*{example*}{Example}  %% unnumbered
\end{verbatim}


Here's how to number by report.
\begin{verbatim}
  \theoremstyle{plain}% default
  \newtheorem{theorem}{Theorem}  %%number by article
  \newtheorem{lemma}[theorem]{Lemma}
  \newtheorem*{corollary*}{Corollary}  %% unnumbered

  \theoremstyle{definition}
  \newtheorem{definition}[theorem]{Definition}
 \newtheorem{problem}[theorem]{Problem}
  \newtheorem{example}[theorem]{Example}
  \newtheorem*{problem*}{Problem}    %% unnumbered
  \newtheorem*{example*}{Example}  %% unnumbered
\end{verbatim}

Here are some examples

%
\begin{verbatim}
  \begin{problem}
  This gives me a normal numbered problem.
  \end{problem}
  \begin{example*}
  This gives me an unnumbered example heading.
  \end{example*}
\end{verbatim}
%
which produces:

  \begin{problem}
  This gives me a normal numbered problem.
  \end{problem}
  \begin{example*}[My heading]
  This gives me an unnumbered but named, example heading.
  \end{example*}
%

\smallskip
In order to allow
authors maximum flexibility in numbering and naming, \emph{no} theorem-like
environments are defined in \verb"MIIR.cls". Rather, you have to define
each one yourself, as above.

\subsection{Proofs}
 \label{proofs}
The \verb"proof" environment is included in the
amsthm package and provides a consistent format for proofs.  An analogous command, \verb"\solution", is included in the MIIR.class file and draws on it.

 For example,
\begin{verbatim}
  \begin{proof}\ 
    Use $K_\lambda$ and $S_\lambda$ to translate combinators
    into $\lambda$-terms. For the converse, translate
    $\lambda x$ \ldots by [$x$] \ldots and use induction
    and the lemma.
  \end{proof}
\end{verbatim}
produces the following:
  \begin{proof}\ 
    Use $K_\lambda$ and $S_\lambda$ to translate combinators
    into $\lambda$-terms. For the converse, translate
    $\lambda x$ \ldots\ by [$x$] \ldots\ and use induction
    and the lemma.
  \end{proof}
  
  Similarly
  \begin{verbatim}
  \solution{
    Use $K_\lambda$ and $S_\lambda$ to translate combinators
    into $\lambda$-terms. For the converse, translate
    $\lambda x$ \ldots\ by [$x$] \ldots\ and use induction
    and the lemma.}
 \end{verbatim}
yields
\solution{
    Use $K_\lambda$ and $S_\lambda$ to translate combinators
    into $\lambda$-terms. For the converse, translate
    $\lambda x$ \ldots\ by [$x$] \ldots\ and use induction
    and the lemma.}
 

\paragraph{Adapting the `Proof' heading}
An optional argument allows you to have a different
name from the simple `Proof'. For example, to change the heading
to read `Proof of the Pythagorean Theorem', key the following:

\begin{verbatim}
  \begin{proof}[Proof of the Pythagorean Theorem]
    Start with a generic right-angled triangle \ldots
  \end{proof}
\end{verbatim}

It produces:
  \begin{proof}[Proof of the Pythagorean Theorem]\ 
    Start with a generic right-angled triangle \ldots
  \end{proof}

\paragraph{Typesetting a Proof or Solution without a \qedsymbol}
Use \verb"proof*" or \verb"solution*" which are not part of the amsthm package. The environments are defined in MIIR.cls.
For example,
\begin{verbatim}
  \begin{solution*}
    The apparent virtual mass coefficient \ldots
  \end{solution*}
\end{verbatim}
produces the following:
  \begin{solution*}
    The apparent virtual mass coefficient \ldots
  \end{solution*}

\section{Displayed equations}

The MIIR class file will insert the correct space above and below
displayed maths if standard \LaTeX\ commands are used; 
\verb"\[ ... \]" are recommended rather than \verb"$$ ... $$". 

Do not leave blank
lines above and below displayed equations unless a new paragraph is
really intended. Punctuate displayed equations.

\paragraph{Numbering of equations}

Equations are numbered in one sequence throughout each section. If you wish to number any equations in a subsequence then use the  subequations feature of the amsmath package, as illustrated below at \eqref{subeq}.

If you really insist on numbering by report then add the following line to your root file.
\begin{verbatim}
\renewcommand{\theequation}\arabic{equation}}
\end{verbatim}
Here are some examples.

\begin{verbatim}
\[
E=Mc^2.
\]
\begin{equation}\label{einstein}
E = Mc^2 .
\end{equation}
Here are examples of a multiline displays:
\begin{equation}
\left.\begin{array}{rcl}
x &=& a+b\\
y &=& c+d\\
z &=& e+f.
\end{array}\right\}
\end{equation}
\renewcommand{\theequation}{\arabic{equation}}
\setcounter{equation}{0}
\begin{eqnarray}
x&=&a+a+b+b\nonumber\\
&=&2a+b+b\nonumber\\
&=&2a+2b.
\end{eqnarray}
\renewcommand{\theequation}{\thesection.\arabic{equation}}
\setcounter{equation}{0}
\begin{equation}\label{einstein2}
E = Mc^2 .
\end{equation}
\begin{subequations}
\begin{align}
  - \nabla p + \mu\nabla^2\mathbf{u} &= 0 , \\
  \nabla\cdot\mathbf{u} &= 0. 
\end{align}
\end{subequations}
\end{verbatim}
\[
E=Mc^2.
\]

\begin{equation}\label{einstein}
E = Mc^2 .
\end{equation}
Here are examples of a multiline displays:
\begin{equation}
\left.\begin{array}{rcl}
x &=& a+b\\
y &=& c+d\\
z &=& e+f.
\end{array}\right\}
\end{equation}
\renewcommand{\theequation}{\arabic{equation}}
\setcounter{equation}{0}
\begin{eqnarray}
x&=&a+a+b+b\nonumber\\
&=&2a+b+b\nonumber\\
&=&2a+2b.
\end{eqnarray}
\renewcommand{\theequation}{\thesection.\arabic{equation}}
\setcounter{equation}{0}
\begin{equation}\label{einstein2}
E = Mc^2 .
\end{equation}
\begin{subequations}
\begin{align}
  - \nabla p + \mu\nabla^2\mathbf{u} &= 0 , \label{subeq}\\
  \nabla\cdot\mathbf{u} &= 0. 
\end{align}
\end{subequations}

\section{Headings}

\LaTeX\ provides five levels of section headings, only four of which are
defined in the MIIR class file:
\begin{itemize}
  \item[] Heading A -- \verb"\section{...}"
  \item[] Heading B -- \verb"\subsection{...}"
  \item[] Heading C -- \verb"\subsubsection{...}"
  \item[] Heading D -- \verb"\paragraph{...}"
\end{itemize}
The \verb"subparagraph" heading is not provided.


\section{Floats}
\subsection{Tables}

The usual \verb"table" environment  produces consecutively numbered tables.

Table captions should appear before the body of the table; therefore you
should place the \verb"\caption" command before the \verb"\begin{tabular}" command.
The \verb"\label" must follow the caption. An optional argument to \verb"\caption" provides the text used in the
\verb"\listoftables", if any.

The MIIR style dictates that vertical rules should never be used within the
body of the table. 

\subsection{Figures}

Include figures using the graphicx.sty package
 by using the following commands

 \begin{verbatim}
\begin{figure}
\includegraphics[]{file}
  \caption[The general picture]{An example figure illustrating the general picture.}
  \label{sample-figure}
\end{figure}
\end{verbatim}

Within the optional argument you can include size and orientation requirements; the main argument will include the file name.
The caption should follow the figure; the label must follow the caption. The optional argument to \verb"\caption" is only needed if a \verb"\listoffigures" is required.  Positioning of the figure within the report is determined by the class file.

\subsection{Landscape}
 Adding

 \begin{verbatim}
  \usepackage[figuresright]{rotating}
  \usepackage{floatpag}
  \rotfloatpagestyle{empty}
  \end{verbatim}
%
will enable wide figures (and tables) to be set as landcape as follows: 

\begin{verbatim}
\begin{sidewaysfigure}
\centering
\includegraphics[]{}
\caption{Landscape figure}
\label{sidefig}
\end{sidewaysfigure}
\end{verbatim}

\section{Acknowledgements}

Acknowledgements should appear at the close of your paper, just before
any appendices and the list of references. Use the \verb"acknowledgement"
or \verb"acknowledgements" environment, which will also typeset the
unnumbered section heading.
\begin{verbatim}
\begin{acknowledgements}
Thanks to A,B,C.
\end{acknowledgements}
\end{verbatim}
produces

\begin{acknowledgements}
Thanks to A,B,C.
\end{acknowledgements}




\section{Appendices}

You should use the standard  \verb"\appendix" command to place any
Appendices -- normally, these are just before the references. This command numbers
appendices as A, B etc., equations as (A1), (B1) etc., Figures and
Tables numbered as A1, B1 and so on. Some examples follow the next section.

\section{References}\label{refs}

Reference lists consist of documents you actually cite in the text; bibliographies
may also list items that are not actually cited so may, for example, contain further reading.
They should be included at the end of the report.

Reference lists can
be created automatically from a bibliographic database, i.e. a \verb".bib" file, or manually; in either case
you will need a style file to interpret the commands properly.
We have chosen the natbib package because of its versatility. Include in the preamble the following command:

\begin{verbatim}
\usepackage[numbers]{natbib} % optional argument for numbered references
\end{verbatim}

Using natbib citation commands will mean your report can be much more easily updated and corrected, and will also mean that links can be included. 
Refer to works cited  in the text by using the usual natbib commands:
 a selection is provided below; there are many more. 

\begin{tabular}{@{}ll@{}}
\verb"\citep{MenshEst}"
    & $\rightarrow\enskip$\citep{MenshEst}\\
\verb"\citep[see][p.$\,$34]{MenshEst}"
    & $\rightarrow\enskip$\citep[see][p.$\,$34]{MenshEst}\\
\verb"\citep[e.g.][]{MenshEst}"
    & $\rightarrow\enskip$\citep[e.g.][]{MenshEst}\\
\verb"\citep[Section~2.3]{MenshEst}"
    & $\rightarrow\enskip$\citep[Section~2.3]{MenshEst}\\
\verb"\citep{AizenBar, MenshEst}"\\
    & $\hspace{-70pt}\rightarrow\enskip$\citep{AizenBar, MenshEst}\\
\verb"\cite{AizenBar, MenshEst,}"\\
    & $\hspace{-70pt}\rightarrow\enskip$\cite{AizenBar, MenshEst}\\
\verb"\cite{MenshEst}"
    & $\rightarrow\enskip$\cite{MenshEst}\\
\verb"\citealp{MenshEst}"
    & $\rightarrow\enskip$\citealp{MenshEst}\\
\end{tabular}

Items will be appeared in a numbered list; these numbers will appear at the position of the citation command.



\subsection{Automatic lists using \textsc{Bib}\upshape{\TeX}}
You will need a \verb".bib" file, and a standard \verb".bst" file that creates a reference
list from that. You can use either the \verb"MIIR.bst"  or \verb"plain.bst".
Executing BibTeX on the .bib file will create a .bbl file in the correct style.



\subsection{Keying in your reference list from a .bbl file}
\label{authordatebiblio}

If you are not constructing a list of references from a database, but still want automatic referencing,
then you will need to create a .bbl file, keying entries as below.


\begin{verbatim}
  \begin{thebibliography}{9}
    \expandafter\ifx\csname natexlab\endcsname\relax
      \def\natexlab#1{#1}\fi
    \expandafter\ifx\csname selectlanguage\endcsname\relax
      \def\selectlanguage#1{\relax}\fi

  %  \bibitem[Aizenman and Barsky, 1987]{AizenBar}
      \bibitem{AizenBar}
Aizenman, M., and Barsky, D.~J. 1987.
    Sharpness of the phase transition in percolation models.
    {\em Comm. Math. Phys.}, {\bf 108}, 489--526.
 
%  \bibitem[Menshikov, 1985]{MenshEst}
  \bibitem{MenshEst}    
Menshikov, M.~V. 1985.
    Estimates for percolation thresholds for lattices in {${\bf R}\sp n$}.
    {\em Dokl. Akad. Nauk SSSR}, {\bf 284}, 36--39.

\end{verbatim}

For numbered references the optional argument to the \verb"\bibitem" command is not required
but is included and commented out for illustrative purposes.


\newcommand\strttab{\noindent\begin{tabular}{@{}p{10pc}@{}p{21pc}@{}}}

\appendix
\section{Special commands in {\mdseries\texttt{MIIR.cls}}}
 \addtocontents{toc}{\setcounter{tocdepth}{1}}

The following is a summary of the new commands, optional
arguments and environments that have been added to the
standard \LaTeX\ user-interface in creating the MIIR class file.
\vspace{6pt}


\strttab
{\em New commands}      & \\[3pt]
\verb"\summary"& A short description of the problem, the method and the conclusion \\
\verb"\solution"  &similar to `Proof' but provides `Solution'.\\
\verb"\affiliation"     & use after \verb"\author" to typeset the author
                          affiliation(s). Do not use a \verb"\\" command
                          in \verb"\author" to start an affiliation (as
                          in the standard \LaTeX\ styles).\\
\verb"\and"             & to typeset `and' before the final author's name.\\
\verb"\ls", \verb"\ns"  & to add letterspacing after authors' names.\\

\verb"\nbcite"       & works in the same way as the normal \verb"\cite"
                       command except it doesn't put in the `[ ]'s.\\
\end{tabular}

\strttab
{\em New environments}  & \\[3pt]
\verb"acknowledg(e)ment(s)"  & to typeset the acknowledgments section.\\
\verb"bottomfigure"     & for split figures and captions (on facing page).\\
\verb"proof*, solution*"           & to typeset mathematical proofs and solutions without the
                           terminating proofbox. \\
\verb"studygroup"  &Study Group information\\
\verb"communicated"  &Study Group organiser who communicated the report\\
\verb"partner" & Name of the industry partner\\
\verb"presenter" & Name(s) of individual(s) from the partner who proposed
the problem\\
\verb"team" & Names of the people who worked on the problem\\
\verb"application" & Application area(s)\\
\verb"tools" & Methods and models used to tackle the problem\\
\verb"keywords" & Keywords to describe the problem\\
\verb"MSC2020" & Two-level MSC2020 codes
\end{tabular}


\strttab
{\em New optional arguments} & \\[3pt]
\verb"[<short title>]"  & in the \verb"\title" command: to define a shorter
                           title to be used in the running head.\\
\verb"[<short author>]" & in the \verb"\author" command: to define a
                           shorter version of the author names for use
                            in the running head.\\
\verb"[<widest label>]" & in \verb"\begin{enumerate}": to ensure the correct
                           alignment of numbered lists with wide
                           labels.
\end{tabular}



\subsection{Catchline commands}

To be placed in the preamble:
\begin{itemize}
  \item \verb"\date{}" -- to set the `Communicated to MIIR' date.
\end{itemize}

\section{Footnotes}

Footnotes are listed by number throughout the report. If a footnote marker falls at the bottom of a typeset page, it is possible for the
footnote text to appear on the next page (a feature of \TeX ). Check
for this.

\section{Fonts}
The default font is Computer Modern. Other fonts can be called through packages such as amsfonts.sty, latexsym.sty, bm.sty.
We recommend keeping things simple.

\subsection{Font sizes}

The MIIR class file defines all the standard \LaTeX\ font sizes. For example:
\begin{itemize}
  \item \verb"\tiny" -- {\tiny This is tiny text.}
  \item \verb"\scriptsize" -- {\scriptsize This is scriptsize text.}
  \item \verb"\footnotesize" -- {\footnotesize This is footnotesize text.}
  \item \verb"\small" -- {\small This is small text.}
  \item \verb"\normalsize" -- This is normalsize text (default).
  \item \verb"\large" -- {\large This is large text.}
  \item \verb"\Large" -- {\Large This is Large text.}
  \item \verb"\LARGE" -- {\LARGE This is LARGE text.}
  \item \verb"\huge" -- {\huge This is huge text.}
  \item \verb"\Huge" -- {\Huge This is HUGE text.}
\end{itemize}
%
MIIR.cls also defines the following new sizes:
\begin{itemize}
  \item \verb"\abstractsize" -- {\abstractsize This is abstractsize text.}
  \item \verb"\authorsize" -- {\authorsize This is authorsize text.}
  \item \verb"\catchlinesize" -- {\catchlinesize This is catchlinesize text.}
\end{itemize}
%
All these sizes are summarized in Table~\ref{tab:fontsizes}.
%
\begin{table}
  % trick to make a ? the width of a number
  \catcode`\?=\active \gdef?{\setbox0=\hbox{0}\hbox to\wd0{}}
  \caption{Type sizes for \LaTeX\ size-changing commands}\label{tab:fontsizes}
  \tabcolsep=7pt
  \begin{tabular}{lcp{70mm}}
    \hline\hline
    Size & Size/Baseline & Usage\\[3pt]
    \verb"\tiny"         &     ?5/6?    & -- \\
    \verb"\scriptsize"   &     ?7/8?    & -- \\
    \verb"\catchlinesize"&     ?8/9?    & title page catchline. \\
    \verb"\authorsize"   &     ?8/10    & author affiliations and received line. \\
    \verb"\small"        &     ?9/10    & -- \\
    \verb"\footnotesize" &     ?9/11    & footnotes, figure captions,
                                          bibliography, tables and quotes/extracts. \\
    \verb"\abstractsize" &     ?9/12    & report summary/abstract. \\
    \verb"\normalsize"   &     10/13    & main text size, A,~B,~C and D headings,
                                          author names and table captions. \\
    \verb"\large"        &     11/13    & part number (parts are not normally used). \\
    \verb"\Large"        &     14/18    & -- \\
    \verb"\LARGE"        &    17/21
                                        & report title and part title. \\
    \verb"\huge"         &     20/25    & -- \\
    \verb"\Huge"         &     25/30    & -- \\
    \hline\hline
  \end{tabular}
\end{table}



   \begin{thebibliography}{9}
    \expandafter\ifx\csname natexlab\endcsname\relax
      \def\natexlab#1{#1}\fi
    \expandafter\ifx\csname selectlanguage\endcsname\relax
      \def\selectlanguage#1{\relax}\fi
      
  %  \bibitem[Aizenman and Barsky, 1987]{AizenBar}
 \bibitem{AizenBar}
Aizenman, M., and Barsky, D.~J. 1987.
    Sharpness of the phase transition in percolation models.
    {\em Comm. Math. Phys.}, {\bf 108}, 489--526.
  
%    \bibitem[Menshikov, 1985]{MenshEst}
\bibitem{MenshEst}
Menshikov, M.~V. 1985.
    Estimates for percolation thresholds for lattices in {${\bf R}\sp n$}.
    {\em Dokl. Akad. Nauk SSSR}, {\bf 284}, 36--39.
  \end{thebibliography}


\label{lastpage}

\end{document}
